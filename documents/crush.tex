\documentclass[12pt,preprint,letterpaper]{aastex}

\begin{document}

\begin{abstract}
There are far more astronomical sources for which we have imaging than spectroscopy;
  this gap between photometry and spectroscopy is expected to grow in the future.
If a set of objects lying at different redshifs,
  but otherwise physically similar,
  are observed photometrically through a set of fixed bandpasses,
  it is in principle possible to infer the object spectral properties at a wavelength resolution
  much higher than the effective resolution implied by the photometric bandpass widths.
Here we demonstrate that this inference of higher-resolution spectroscopy
  from lower-resolution photometry
  is possible in a set of real observations of massive galaxies at redshifts around two
  taken from the XXX Survey.
We show that we can infer both the spectra and the individual object redshifts simultaneously,
  and that the quality of the inference is limited by the signal-to-noise ratio in the imaging.
We also show that it is possible to infer not just the mean spectra for a group of objects presumed similar,
  but also a low-dimensional representation of the intrinsic variance around that mean
  (similar to that returned by principal components analysis).
For the massive galaxies at redshift two we find XXX.
\end{abstract}

~~

\end{document}
