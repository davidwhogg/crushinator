% This document is part of the crushinator project.
% Copyright 2013 the authors.

\documentclass[12pt,preprint,letterpaper]{aastex}

\newcommand{\dd}{\mathrm{d}}

\begin{document}

\author{by Ross~Fadely, David~W.~Hogg, and others}

\begin{abstract}
There are far more astronomical sources for which we have imaging than spectroscopy;
  this gap between photometry and spectroscopy is expected to grow in the future.
If a set of objects lying at different redshifts,
  but otherwise physically similar,
  are observed photometrically through a set of fixed bandpasses,
  it is in principle possible to infer the object spectral properties at a wavelength resolution
  much higher than the effective resolution implied by the photometric bandpass widths.
Here we demonstrate that this inference of higher-resolution spectroscopy
  from lower-resolution photometry
  is possible in a set of real observations of massive galaxies at redshifts around two
  taken from the XXX Survey.
We show that we can infer both the spectra and the individual object redshifts simultaneously,
  and that the quality of the inference is limited by the signal-to-noise ratio in the imaging.
We also show that it is possible to infer not just the mean spectra for a group of objects presumed similar,
  but also a low-dimensional representation of the intrinsic variance around that mean
  (similar to that returned by principal components analysis).
For the massive galaxies at redshift two we find XXX.
\end{abstract}

\section{Introduction}

...Repeat the stuff that Vy showed us...

...Explain why we might be able to work at higher resolution...

...Under what crazy circumstances can we even \emph{think} about doing this?..

\section{Model and method}

There are $N$ sources $n$, each of which may have been observed photometrically through each of $X$ bandpasses $x$.
Each bandpass $x$ is defined by a full-system throughput curve $R_x(\lambda)$,
  which gives the expected contribution to the total detector signal from a photon of wavelength $\lambda$
  impinging on the top of the atmosphere (and headed into the telescope aperture),
  when band $x$ is being observed.
In the simplified case of a photon counting device (like a CCD),
  the function $R_x(\lambda)$ gives the probability that a top-of-atmosphere photon gets detected.
For a bolometric device, the interpretation is slightly different.

The model is that each of these sources $n$ has a true spectrum $f_n(\lambda)$ (energy per unit time per unit area per unit wavelength)
  that is some amplitude $a_n$ times a normalized ``template'' spectrum $g(\lambda)$.
Each source $n$ also has a redshift (Doppler Shift) $z_n$.
The true spectrum template $g(\lambda)$, the true amplitude $a_n$, and the true redshift $z_n$ are all latent variables,
  although in some cases we will have strong prior information about the redshifts.

For reasons given elsewhere (CITE HOGG \& BALDRY), the expectation value for the magnitude $m_{xn}$
  of source $n$ observed in bandpass $x$ is given by
\begin{eqnarray}
E[m_{xn}] &=&
  -2.5\,\log_{10} q_{xn}
\\
q_{xn} &=&
  \frac{\displaystyle\int\frac{a_n}{1+z_n}\,g(\frac{\lambda}{1+z_n})\,R_x(\lambda)\,\lambda\,\dd\lambda}%
       {\displaystyle\int f_0(\lambda)\,R_x(\lambda)\,\lambda\,\dd\lambda}%
\quad ,
\end{eqnarray}
where the $[1+z_n]^{-1}$ factor is a convention to make the $a_n$ somewhat more interpretable as amplitudes,
  the integrals contain an extra factor of $\lambda$ because $R_x(\lambda)$ is defined as the mean \emph{per-photon} signal contribution,
  and $f_0(\lambda)$ is the spectrum of the zero-magnitude absolute calibration source.
This $f_0(\lambda)$ is the spectrum of Vega for Vega magnitudes,
  and [SOMETHING HERE] for AB magnitudes.

...work in expected maggies $q_{xn}$ where uncertainties are likely to be close to Gaussian, even for faint sources...

...parameterize $g(\lambda)$ as a set of points, evenly spaced in $\ln\lambda$.  Alternatives to this?.. interpolation and edge issues?..

...write down likelihood...

...write down smoothness prior or regularization...

\end{document}
